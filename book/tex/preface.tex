\preface

Welcome to the second edition of ``ggplot2: elegant graphics for data
analysis''. I'm so excited to have an updated book that shows off all
the latest and greatest ggplot2 features, as well as the great things
that have been happening in R and in the ggplot2 community the last five
years. The ggplot2 community is vibrant: the ggplot2 mailing list has
over 7,000 members and there is a very active Stack Overflow community,
with nearly 10,000 questions tagged with ggplot2. While most of my
development effort is no longer going into ggplot2 (more on that below),
there's never been a better time to learn it and use it.

I am tremendously grateful for the success of ggplot2. It's one of the
most commonly downloaded R packages (over a million downloads in the
last year!) and has influenced the design of graphics packages for other
languages. Personally, ggplot2 has bought me many exciting opportunities
to travel the world and meet interesting people. I love hearing how
people are using R and ggplot2 to understand the data that they care
about.

A big thanks for this edition goes to Carson Sievert, who helped me
modernise the code, including converting the sources to R Markdown. He
also updated many of the examples and helped me proofread the book.

\section*{Major changes}

I've spent a lot of effort ensuring that this edition is a true upgrade
over the first. As well as updating the code everywhere to make sure
it's fully compatible with the latest version of ggplot2, I have:

\begin{itemize}
\item
  Shown much more code in the book, so it's easier to use as a
  reference. Overall the book has a more ``knitr''-ish sensibility:
  there are fewer floating figures and tables, and more inline code.
  This makes the layout a little less pretty but keeps related items
  closer together.
\item
  Published the complete source online at
  \url{https://github.com/hadley/ggplot2-book}.
\item
  Switched from \texttt{qplot()} to \texttt{ggplot()} in the
  introduction, \protect\hyperlink{cha:getting-started}{intro}. Feedback
  indicated that \texttt{qplot()} was a crutch: it makes simple plots a
  little easier, but it doesn't help with mastering the grammar.
\item
  Added practice exercises throughout the book so you can practice new
  techniques immediately after learning about them.
\item
  Added pointers to the rich ecosystem of packages that have built up
  around ggplot2. You'll now see a number of other packages highlighted
  in the book, and get pointers to other packages I think are
  particularly useful.
\item
  Overhauled the toolbox chapter,
  \protect\hyperlink{cha:toolbox}{toolbox}, to cover all the new geoms.
  I've added a completely new section on text labels,
  \protect\hyperlink{sec:labelling}{labels}, since it's important and
  not covered in detail elsewhere. The mapping section,
  \protect\hyperlink{sec:maps}{maps}, has been considerably expanded to
  talk more about the different types of map data, and where you might
  find them.
\item
  Completely rewritten the scales chapter,
  \protect\hyperlink{cha:scales}{scales}, to focus on the most important
  tasks. It also discusses the new features that give finer control over
  legend appearance, \protect\hyperlink{sec:legends}{legends}, and shows
  off some of the new scales added to ggplot2,
  \protect\hyperlink{sec:scale-details}{scales}.
\item
  Split the data analysis chapter into three pieces: data tidying (with
  tidyr), \protect\hyperlink{cha:data}{tidyr}; data manipulation (with
  dplyr), \protect\hyperlink{cha:dplyr}{dplyr}; and model visualisation
  (with broom), \protect\hyperlink{cha:modelling}{models}. I discuss the
  latest iteration of my data manipulation tools, and introduce the
  fantastic broom package by David Robinson.
\end{itemize}

The book is accompanied by a new version of ggplot2: version 2.0.0. This
includes a number of minor tweaks and improvements, and considerable
improvements to the documentation. Coming back to ggplot2 development
after a considerable pause has helped me to see many problems that
previously escaped notice. ggplot2 2.0.0 (finally!) contains an official
extension mechanism so that others can contribute new ggplot2 components
in their own packages. This is documented in a new vignette,
\texttt{vignette("extending-ggplot2")}.

\section*{The future}

ggplot2 is now stable, and is unlikely to change much in the future.
There will be bug fixes and there may be new geoms, but there will be no
large changes to how ggplot2 works. The next iteration of ggplot2 is
ggvis. ggvis is significantly more ambitious because it aims to provide
a grammar of \emph{interactive} graphics. ggvis is still young, and
lacks many of the features of ggplot2 (most notably it currently lacks
facetting and has no way to make static graphics), but over the coming
years the goal is to make ggvis better than ggplot2.

The syntax of ggvis is a little different to ggplot2. You won't be able
to trivially convert your ggplot2 plots to ggvis, but we think the cost
is worth it: the new syntax is considerably more consistent, and will be
easier for newcomers to learn. If you've mastered ggplot2, you'll find
your skills transfer very well to ggvis and after struggling with the
syntax for a while, it will start to feel quite natural. The important
skills you learn when mastering ggplot2 are not the programmatic details
of describing a plot in code, but the much harder challenge of thinking
about how to turn data into effective visualisations.

\section*{Acknowledgements}

Many people have contributed to this book with high-level structural
insights, spelling and grammar corrections and bug reports. I'd
particularly like to thank William E. J. Doane, Alexander Forrence,
Devin Pastoor, David Robinson, and Guangchuang Yu, for their detailed
technical reviews of the book.

Many others have contributed over the (now quite long!) lifetime of
ggplot2. I would like to thank: Leland Wilkinson, for discussions and
comments that cemented my understanding of the grammar; Gabor
Grothendieck, for early helpful comments; Heike Hofmann and Di Cook, for
being great advisors and supporting the development of ggplot2 during my
PhD; Charlotte Wickham; the students of stat480 and stat503 at ISU, for
trying it out when it was very young; Debby Swayne, for masses of
helpful feedback and advice; Bob Muenchen, Reinhold Kliegl, Philipp
Pagel, Richard Stahlhut, Baptiste Auguie, Jean-Olivier Irisson, Thierry
Onkelinx and the many others who have read draft versions of the book
and given me feedback; and last, but not least, the members of R-help
and the ggplot2 mailing list, for providing the many interesting and
challenging graphics problems that have helped motivate this book.

\vspace{\baselineskip}
\begin{flushright}\noindent
{\it Hadley Wickham}\\
September 2015\\
\end{flushright}
